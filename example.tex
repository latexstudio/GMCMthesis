% !Mode:: "TeX:UTF-8"
%!TEX program  = xelatex

\documentclass[bwprint]{gmcmthesis}
% \documentclass[withoutpreface,bwprint]{cumcmthesis}
% 去掉封面与编号页

\title{基于监控视频的前景目标提取}
\tihao{A}
\baominghao{10351013} %参赛队号
\schoolname{温州大学}%学校名称
\membera{杨超} %队员A
\memberb{刘新伟} %队员B
\memberc{张磊} %队员C
\begin{document}
 \maketitle
 \begin{abstract}
 本题主要是围绕视频监控开展的一系列问题。为了从视频监控中获取有效信息,我们需要解决前景目标提取、抗干扰、提取关键帧以及事件检测等一系列问题。问题设置层层递进,问题一需要解决无晃动静态背景的前景目标提取,问题二在问题一基础上需要解决无晃动动态背景的前景目标提取,问题三又在问题二的基础上增加新的要求,需要将有晃动的视频进行修正,而问题四则是从一段视频中提取包含前景目标的关键帧,问题五则是在问题四的基础上运用同步对应方法多角度分析相关性信息,问题六便是在问题五的基础上进行事件检测来应对现实生活中的一些紧急问题。根据这些特点对问题1 用背景减除的方法解决;对问题2 用Vibe与背景减除法结合的方法解决;对问题3 用特征窗追踪的方法解决;对问题4用帧差欧氏距离法解决;对问题5用同步对应法解决;对问题六用光流法解决。
对于问题1,用图像差分法中的静态背景建模法建立了 背景减除算法模型。,然后借助于背景减除法和Matlab软件,对附件中所提供的数据进行了处理,去除异常数据,对残缺数据进行适当补充,并从中选取了1个典型无晃动静态背景视频对理论结果进行了前景提取,输出结果显示,理论结果与实际操作结果吻合。
对于问题2,为了解决无晃动动态背景下前景提取问题,在对背景减除算法模型改进的基础上结合Vibe前景提取算法设计了“Vibe+背景减除+迭代”的运动前景提取方法。对模型进行了合理的理论证明和推导,并对视频数据进行了试验,证明通过迭代的方法将Vibe和背景减除法相结合,能够很好的消除上面所提到的噪音。
对于问题3,在前两问的基础上添加了基于特征窗追踪的稳像算法来解决有晃动的前景提取问题。通过算法修正有晃动视频,再通过前两问的方法解决这类问题。
对于问题4,我们通过帧差欧式距离法提取各个子镜头的关键帧,并在此基础上,运用 同步对应方法进行目标可见性判断,检测并提取问题5中的视频前景目标。
对于问题5,我们通过一种可靠性的目标交接最佳时机的判断方法,对前景目标所处的区域进行区域判断,进而实现对前景目标的有效检测和提取。
对于问题6,我们用光流法通过查看光点的移动速率和所处的位置来甄别是否有人群惊慌逃散,是否有人群短期聚集,或者做一些有规则的动作。 
 


\keywords{前景提取;背景减除法; Vibe前景提取法;  特征追踪法; 稳像算法; 欧氏距离法;同步对应法; 光流法}
\end{abstract}

\pagestyle{plain}
\section{问题重述}

\subsection{问题的背景}

视频监控是中国安防产业中最为重要的信息获取手段。随着“平安城市”建设的顺利开展,各地普遍安装监控摄像头,利用大范围监控视频的信息,应对安防等领域存在的问题。近年来,中国各省市县乡的摄像头数目呈现井喷式增长,大量企业、部门甚至实现了监控视频的全方位覆盖。如北京、上海、杭州监控摄像头分布密度约分别为71、158、130个/平方公里,摄像头数量分别达到115万、100万、40万,为我们提供了丰富、海量的监控视频信息。
目前,监控视频信息的自动处理与预测在信息科学、计算机视觉、机器学习、模式识别等多个领域中受到极大的关注。而如何有效、快速抽取出监控视频中的前景目标信息,是其中非常重要而基础的问题[1-6]。这一问题的难度在于,需要有效分离出移动前景目标的视频往往具有复杂、多变、动态的背景[7,8]。这一技术往往能够对一般的视频处理任务提供有效的辅助。以筛选与跟踪夜晚时罪犯这一应用为例:若能够预先提取视频前景目标,判断出哪些视频并未包含移动前景目标,并事先从公安人员的辨识范围中排除;而对于剩下包含了移动目标的视频,只需辨识排除了背景干扰的纯粹前景,对比度显著,肉眼更易辨识。因此,这一技术已被广泛应用于视频目标追踪,城市交通检测,长时场景监测,视频动作捕捉,视频压缩等应用中。


\subsection{问题的提出}

问题1:对含有静态背景和摄像头稳定拍摄时间大约5秒的监控视频,构造提取前景目标(如人、车、动物等)的数学模型,并针对这个模型设计有效的求解方法。(附件2中含有这类特征的典型监控视频) 
问题2:对含有动态背景信息的监控视频,设计有效的方案提取前景目标。(附件2中含有这类特征的典型监控视频)
问题3:监控视频中,当监控摄像头出现晃动或偏移的现象时,视频也会发生短暂的抖动现象(这种视频变换在短时间内可可近似看作为一种线性仿射变换,如旋转、平移、尺度变化等)。怎样有效地提取这类视频的前景目标? 
问题4:用所构造的建模方法,从附件3中的每组视频中选出含有显著前景目标的视频帧标号,将其独立成段表示在建模论文正文中。并且需注明前景目标是出现于哪个视频的哪些帧。
问题5:怎样从不同角度同时拍摄的近似为同一地点的多个监控视频中有效检测和提取视频前景目标?充分利用多角度视频的前景之间的相关性信息
问题6:可否根据获取的前景目标信息,自动判断监控视频中有无人群短时聚集、人群惊慌逃散、群体规律性变化(如跳舞、列队排练等)、物体爆炸、建筑物倒塌等异常事件?可以考虑包括前景目标奔跑的线性变化形态特征、前景规律性变化的周期性特征等特征信息。尝试对更多的异常事件类型,设计相应的事件检测方案。


\section{模型的假设}

\begin{itemize}
\item 题中所给的视频数据均可靠无误;
\item  动态背景中干扰因素是有一定规律性的;
\item  有晃动视频中前景提取目标不恒在晃动区域边缘;
\item  不同角度同时拍摄的内容为同一地点;
\item  帧率相同时,不同角度拍摄的下多个监控视频帧数相同。
  
\end{itemize}

\section{符号说明}

\begin{tabular}{cc}
 \hline
 \makebox[0.4\textwidth][c]{符号}	&  \makebox[0.5\textwidth][c]{意义} \\ \hline
 $D_k$	    & 欧氏距离 \\ \hline
 $P(x,y)$    & 某一像素点  \\ \hline
 $f_i$	    & 当前帧  \\ \hline
 $N$	    & 蒙板  \\ \hline
 $T$	    & 映射关系  \\ \hline
 $H$	    & 镜头帧图像数目  \\ \hline
 $R$	    & 第$i$条视野分界线  \\ \hline
 $R$	    & 桌子直径(cm)  \\ \hline
\end{tabular}

\section{问题分析}

\subsection{问题一分析}
本题要解决的是静态背景前景提取问题。在很多的监控场景中,由于前景背景差异小、运动形变,背景里有快速明暗变化,背景中存在几何变化以及光 照等噪声,诸多因素对前景提取效果和质量造成了一定的影响,甚至无法有效地提取到完整前景帧序列。完整的提取前景,也就是不多提取也不少提取,已经成为研究热点和难点。目前,常用的运动前景检测和提取方法有光流法、帧差法、背景减除法等。运动前景提取主要分为以下几大类,均值方差法(背景减除法和帧差法等)、基于模糊的方法 (模糊高斯等)、基于高斯统计的方法(简单高斯,混合高斯等)、基于颜色和纹理特征的统计方法(GMG等)、模糊神经网络法 (模糊自适应神经网络法)。 帧差法实现简单,但对运动变化过快或过慢比较敏感, 容易产生内部空洞;光流法计算比较耗时,难以满足实时提取的要求;背景减除法在有较好的静态背景图像的基础上会有比较好的提取效果。


\subsection{问题二分析}
本题要解决的是动态背景前景提取问题。Olivier Barnich提出了一种称为Vibe的前景提取算法,该算法采用像素点的八邻域像素来创建背景模型(一种背景建模法),通过比对背景模型和当前帧画面的像素值,即可快速地提取并分离出运动前景物体的前景像素。在问题一静态背景前景提取问题的基础上,我们结合了Vibe前景提取法和背景减除法,提出了一种适用于监控场景的迭代算法,提升了运动前景提取的准确性和完整性,并保持了较好的计算性能。

\subsection{问题三分析}
本题要解决的是镜头修正问题。很多时候,摄像机的意外晃动会产生抖动的视频序列,使观看者产生头晕等不适,对于目标自动识别系统会导致漏警和虚警。图像稳定技术就是用来补偿抖动获得图像序列稳定显示的技术,现已广泛应用于军事,工业,民用等各个方面,是计算机视觉领域的研究热点。
    关于稳像问题,研究人员提出了许多算法,其中特征跟踪法受到广泛关注。特征跟踪法的基本思想是通过跟踪特征摄像机运动参数,进行运动补偿,从而得到稳定的显示。该方法的难点在于特征的选取和跟踪算法的精度,不理想的特征检测和跟踪算法都无法得到好的稳像效果。为此,5.3提出一种基于角点检测和特征窗跟踪的稳像算法。该方法以参考帧中检测到的角点为特征,在后续帧中基于模糊Kalman滤波实现特征窗的跟踪,同时利用X84规则进行特征窗的优选,最后综合对应特征窗的运动信息估计摄像机的运动参数,从而利用运动补偿技术将后续帧向参考帧对准,消除视频序列图像间的抖动。

\subsection{问题四分析}
本题要解决的是关键帧提取问题。视频关键帧提取算法的实现在镜头中选择一个或几个具有显著前景目标的帧作为“代表”来描述和表示镜头内容,即使用一定的视频关键帧提取算法提取出视频的若干关键帧。
一个视频是由许多镜头组成的,而镜头是视频的一个物理单元,每个镜头包括一个帧序列,这些帧被连续地记录且表达了一个在时间上和空间上连续的动作,其边界可以通过编辑点或者判断摄像机何时开机或关机决定。镜头之间的边界或者切换类型有很多,最简单的切换是突变(Cut)和渐变(GT)两种类型,其中突变指的是在两个连续的帧之间发生了突然的镜头改变,渐变比突变复杂一些,包含溶解、淡入、淡出、划变等类型。
在进行基于内容的视频检索的过程中,首先必须找出视频序列中发生镜头变换的位置,从而进一步将视频分割成独立的镜头片段,这个过程称为镜头边界检测。镜头边界检测依赖于在边界两侧相邻的两个帧图像之间显示出了内容上的重大改变,需要用一个测量方法来定量地确定在这样的一对帧之间的差别,若这个差别超过了给定的阈值,则表示这是一个镜头边界。对于一个鲁棒的镜头边界检测算法来说,应该能够尽可能高精度地检测出所有这些不同的边界,不管是突变的还是渐变的情况。
因为镜头渐变的检测比突变检测复杂很多,突变检测最常用的度量标准是两个连续帧的直方图之间的差值,目前对视频渐变的检测效果一直不是太好,而已有的镜头边界检测算法大多数是有关视频突变的检测的,典型方法有基于直方图的方法、基于边缘的方法、基于模型的方法、光流检测法等。

\subsection{问题五分析}
对多角度近似同一场景的视频,其多个视频的交集所形成的重叠区域中含有前景目标,利用相关的方法先对重叠区域中的前景目标进行判断,进而完成对前景目标的有效检测和提取。

\subsection{问题六分析}
光流法可以通过查看光点的移动速率和所处的位置来甄别是否有人群惊慌逃散,是否有人群短期聚集,或者做一些有规则的动作。由于目标对象或者摄像机的移动造成的图像对象在连续两帧图像中的移动 被称为光流。它是一个2D 向量场,可以用来显示一个点从第一帧图像到第二帧图像之间的移动。




问题三流程图:
\begin{figure}[!h]
\centering
\includegraphics[width=\textwidth]{1.png}
\caption{问题三流程图}
\end{figure}

%参考文献
\begin{thebibliography}{9}%宽度9
 \bibitem{bib:one} ....
 \bibitem{bib:two} ....
\end{thebibliography}

\newpage
%附录
\appendix
\section{我的 MATLAB 源程序}
\begin{lstlisting}[language=Matlab]%设置不同语言即可。
kk=2;[mdd,ndd]=size(dd);
while ~isempty(V)
[tmpd,j]=min(W(i,V));tmpj=V(j);
for k=2:ndd
[tmp1,jj]=min(dd(1,k)+W(dd(2,k),V));
tmp2=V(jj);tt(k-1,:)=[tmp1,tmp2,jj];
end
tmp=[tmpd,tmpj,j;tt];[tmp3,tmp4]=min(tmp(:,1));
if tmp3==tmpd, ss(1:2,kk)=[i;tmp(tmp4,2)];
else,tmp5=find(ss(:,tmp4)~=0);tmp6=length(tmp5);
if dd(2,tmp4)==ss(tmp6,tmp4)
ss(1:tmp6+1,kk)=[ss(tmp5,tmp4);tmp(tmp4,2)];
else, ss(1:3,kk)=[i;dd(2,tmp4);tmp(tmp4,2)];
end;end
dd=[dd,[tmp3;tmp(tmp4,2)]];V(tmp(tmp4,3))=[];
[mdd,ndd]=size(dd);kk=kk+1;
end; S=ss; D=dd(1,:);


 \end{lstlisting}


\end{document} 